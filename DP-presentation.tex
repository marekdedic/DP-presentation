\documentclass[10pt]{beamer}

\usepackage{polyglossia}
\usepackage{csquotes}
\usepackage{fontspec}
\usepackage{microtype}
\usepackage{color}
\usepackage{url}
\usepackage{hyperref}
\usepackage{amsfonts}
\usepackage{amsmath}
\usepackage{amsthm}
\usepackage{subcaption}
\usepackage[backend=biber,style=iso-authoryear,sortlocale=en_US,autolang=other,bibencoding=UTF8]{biblatex}

\addbibresource{zotero.bib}

\setdefaultlanguage{czech}
\setmainfont{TeX Gyre Termes}
\usetheme{Boadilla}
\usecolortheme{crane}
\setbeamertemplate{section in toc}[ball unnumbered]
\setbeamertemplate{bibliography item}{}

\hypersetup{
	pdfencoding=auto,
	unicode=true,
	citecolor=green,
	filecolor=blue,
	linkcolor=red,
	urlcolor=blue
}

\makeatletter
\newcommand*{\currentSection}{\@currentlabelname}
\makeatother

\newcommand{\mathmat}{\ensuremath{\mathbf}}

\title[Optimalizace vzdálenosti pro multi-instanční shlukovací problémy]
{
	Optimalizace vzdálenosti pro multi-instanční shlukovací problémy
}

\titlegraphic
{
	\includegraphics[width=0.12\columnwidth]{images/fjfi.png}
}

\date{5. února 2020}

\author[Marek Dědič]
{
	Marek~Dědič\inst{1}\inst{2} \\
	Školitel:~Doc.~Ing.~Tomáš~Pevný,~Ph.D.\inst{3}\inst{4}
}

\institute[FJFI ČVUT v Praze]
{
	\inst{1} ČVUT v Praze, Fakulta jaderná a fyzikálně inženýrská, Matematická informatika \and
	\inst{2} Cisco Systems Inc., Karlovo náměstí 10, Praha 2 \and
	\inst{3} ČVUT v Praze, Fakulta elektrotechnická \and
	\inst{4} Avast Software s.r.o., Pikrtova 1737/1a, Praha 4
}

\AtBeginSection[]{
	\begin{frame}{\currentSection}
		\tableofcontents[currentsection]
	\end{frame}
}

\begin{document}

\begin{frame}
	\titlepage
\end{frame}

% Body

\section{Motivace \& Úloha}

\begin{frame}{Motivace}
	\begin{itemize}
		\item Shluková analýza je jedním z nejzákladnějších problémů učení bez učitele.
		\item Klíčovou složkou návrhu shlukovacího algoritmu je volba vzdálenostní metriky a pro složitější úlohy je často obtížné najít správnou metriku.
		\item Multi-instanční učení je způsobem, jak efektivně konstruovat komplexní hierarchické modely cílené na specifickou úlohu.
		\item V diplomové práci byly představeny a srovnány 3 přístupy multi-instanční shlukové analýzy.
	\end{itemize}
\end{frame}

\begin{frame}{Úloha}
	\begin{itemize}
		\item Síťová bezpečnost je atraktivní aplikací navržených metod.
		\item Cílem práce bylo umožnit shlukování domén 2. řádu podle aktivity klientů, kteří se na ně připojují.
		\item Úspěšné shlukování může výrazně podpořit práci síťových analytiků.
		\item Možné budoucí rozšíření na šifrovaná data se stoupajícím šifrováním základních protokolů.
	\end{itemize}
\end{frame}

\begin{frame}{Obsah}
	\tableofcontents
\end{frame}

\section{MIL}

\begin{frame}[c]\frametitle{End--to--end učení}
	\centering
	\includegraphics{images/end_to_end_learning/end_to_end_learning.pdf}
\end{frame}

\begin{frame}[c]\frametitle{Multi instanční učení}
	\centering
	\includegraphics{images/multi_instance_learning/multi_instance_learning.pdf}
\end{frame}

\begin{frame}[c]\frametitle{Paradigma vloženého prostoru}
	\centering
	\includegraphics{images/embedded_space_paradigm/embedded_space_paradigm.pdf}
\end{frame}

\begin{frame}{Vkládající funkce \( \phi \)}
	\centering
	\includegraphics[width=0.9\pagewidth]{images/embedding_function/embedding_function.pdf}
\end{frame}

\section{Contrastive predictive coding}

\begin{frame}{Contrastive predictive coding}
	Vyjádření přístupu pomocí constractive predictive coding jako ztrátové funkce
	\[ \mathmat{D}_{ij} = \left\lVert \phi \left( B_i^{(1)} \right) - \phi \left( B_j^{(2)} \right) \right\rVert_2^2 \]
	\[ L_\mathrm{CPC} = \frac{1}{n} \sum_{i = 1}^n \left( \log \left( \mathmat{D}_{ii} \right) - \log \sum_{\substack{j = 1 \\ j \neq i}}^n \mathmat{D}_{ij} \right) \]
\end{frame}

\section{Triplet loss}

\begin{frame}{Triplet loss}
	Triplet loss je alternativou vyžadující supervised přístup
	\[ \mathmat{y}_{ij} =
\begin{cases}
	1 &\text{pro} \quad y_i = y_j \\
	0 &\text{jinak}
\end{cases}
\]
\[ \mathmat{\eta}_{ij} = \begin{cases}
		1 &\text{pokud taška } B_j \text{ je cílovým sousedem tašky } B_i \\
		0 &\text{jinak}
  \end{cases} \]
\[ L_\mathrm{triplet} = \sum_{ij} \eta_{ij} \mathmat{D}_{ij} + c \sum_{ijl} \eta_{ij} \left( 1 - \mathmat{y}_{il} \right) \left\{ \mathmat{D}_{il} - \mathmat{D}_{ij} \right\}_+ \]
\end{frame}

\section{Magnet loss}

\begin{frame}
	\begin{figure}[h]
		\centering
		\begin{subfigure}[b]{0.18\textwidth}
			\centering
			\includegraphics[width=\textwidth]{images/triplet-magnet-difference/triplet_before.pdf}
			\caption{Triplet loss: před}
		\end{subfigure}
		\hfill
		\begin{subfigure}[b]{0.18\textwidth}
			\centering
			\includegraphics[width=\textwidth]{images/triplet-magnet-difference/triplet_after.pdf}
			\caption{Triplet loss: po}
		\end{subfigure}
		\hfill
		\begin{subfigure}[b]{0.18\textwidth}
			\centering
			\includegraphics[width=\textwidth]{images/triplet-magnet-difference/magnet_before.pdf}
			\caption{Magnet loss: před}
		\end{subfigure}
		\hfill
		\begin{subfigure}[b]{0.18\textwidth}
			\centering
			\includegraphics[width=\textwidth]{images/triplet-magnet-difference/magnet_after.pdf}
			\caption{Magnet loss: po}
		\end{subfigure}
		\caption{Vizualizace rozdílu mezi triplet loss a magnet loss. Obrázek z \cite{rippel_metric_2015}}
	\end{figure}
\end{frame}

\section{Závěr}

\begin{frame}{Závěr}
	\begin{itemize}
		\item Vyzkoušeno několik přístuplů ke shlukování dat
		\item Aplikováno multi-instanční učení na problém shlukování dat
		\item Porovnání přístupů na veřejně dostupných standardních datasetech.
	\end{itemize}
\end{frame}

\end{document}
